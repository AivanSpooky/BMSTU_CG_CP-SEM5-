\renewcommand{\bibname}{}
\begin{thebibliography}{6}
\renewcommand{\bibname}{СПИСОК ИСПОЛЬЗОВАННЫХ ИСТОЧНИКОВ}
\makeatletter
\renewcommand{\@biblabel}[1]{#1.}
\makeatother
\begin{center}
    \textbf{\bibname}
\end{center}
\addcontentsline{toc}{chapter}{СПИСОК ИСПОЛЬЗОВАННЫХ ИСТОЧНИКОВ}
    \bibitem{bib0}
     Порев В. Н. Компьютерная графика //СПб.: БХВ-Петербург. – 2002. – Т. 432. – С. 3.
	\bibitem{bib1}
	Роджерс Д. Алгоритмические основы машинной графики. – Рипол Классик, 1989г.
    \bibitem{bib2}
    Новиков И. Е. Сравнение двух алгоритмов генерации мягких теней //Программа и тез. докл. IX Всерос. конф. молодых ученых по мат. моделированию и информ. технологиям. Кемерово. – 2008г. – С. 28-30.
    \bibitem{bib3}
    Сафина Д. Н. Исследование методов закрашивания Гуро и Фонга. – 2021г.
    \bibitem{bib4}
    Чернявская А. Э. Простые модели освещения 3D-объектов. Особенности цифрового моделирования света //Современные вопросы науки и практики 3. – 2021г. – С. 44.
    \bibitem{bib5}
    Бистерфельд О. А. Методология функционального моделирования IDEF0. – 2013г.
    \bibitem{bib6}
    Hejlsberg A., Wiltamuth S., Golde P. C# language specification. – Addison-Wesley Longman Publishing Co., Inc., 2003г.
    \bibitem{bib7}
    MacDonald M. User Interfaces in C#: Windows Forms and Custom Controls. – Apress, 2008г.
    \bibitem{bib8}
    Windows 11 Home [Электронный ресурс]. URL:
    \url{https://www.officepakke.dk/products/windows-11-home} (дата обращения: 04.11.2024).
    \bibitem{bib9}
    Intel® Core™ i5-10300H Processor [Электронный ресурс]. URL: \url{https://ark.intel.com/content/www/us/en/ark/products/201839/intel-core-i5-10300h-processor-8m-cache-up-to-4-50-ghz.html} (дата обращения: 04.11.2024).
    \bibitem{bib10}
    Stopwatch Класс (System.Diagnostics) [Электронный ресурс]. URL: \url{https://learn.microsoft.com/ru-ru/dotnet/api/system.diagnostics.stopwatch?view=net-8.0} (дата обращения: 06.11.2024).
\end{thebibliography}