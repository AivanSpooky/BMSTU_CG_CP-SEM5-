\begin{center}
    \textbf{ВВЕДЕНИЕ}
\end{center}
\addcontentsline{toc}{chapter}{ВВЕДЕНИЕ}

В современном мире компьютерная графика и моделирование активно применяются в различных областях, таких как разработка игр, архитектурное проектирование и научные исследования. Одной из актуальных задач является моделирование физических объектов и их взаимодействия с окружающей средой в виртуальном пространстве.

\textbf{Цель курсовой работы} $-$ разработка программного обеспечения для моделирования прямоугольной площадки с лунками, соответствующими трехмерным телам (сфера, куб, параллелепипед, шестигранная призма) и возможностью генерации тел с их падением на площадку. Для достижения поставленной цели необходимо решить следующие задачи:

\begin{itemize}
	\item[$-$] описать список доступных к размещению на сцене объектов, формализовать эти объекты;
	\item[$-$] выбрать алгоритмы компьютерной графики для визуализации сцены и объектов на ней;
	\item[$-$] выбрать язык программирования и среду разработки;
    \item[$-$] разработать программное обеспечение и реализовать выбранные алгоритмы визуализации;
    \item[$-$] провести замеры временных характеристик разработанного программного обеспечения.
\end{itemize}