\usepackage{cmap} % Улучшенный поиск русских слов в PDF
\usepackage[T2A]{fontenc} % Поддержка русских букв
\usepackage[utf8]{inputenc} % Кодировка UTF-8
\usepackage[english,russian]{babel} % Языки: русский, английский
\usepackage{ulem} % Для подчёркивания текста
\usepackage{svg}

\usepackage[14pt]{extsizes} % Шрифт 14 по ГОСТ 7.32-2001

\usepackage{xcolor} % Цвета
\usepackage{graphicx} % Работа с изображениями
\usepackage{pgfplots} % Графики
\usetikzlibrary{datavisualization}
\usetikzlibrary{datavisualization.formats.functions}

\newcommand{\img}[3]{ % Команда для вставки изображений
    \begin{figure}[h!]
        \centering
        \includegraphics[height=#1]{img/#2}
        \caption{#3}
        \label{img:#2}
    \end{figure}
}

\usepackage[justification=centering]{caption} % Настройка подписей к рисункам и таблицам
\captionsetup{labelsep=endash} % Разделитель между номером и текстом
\captionsetup[figure]{name={Рисунок}} % Переименование "Figure" на "Рисунок"
\captionsetup[table]{singlelinecheck=false, labelsep=endash} % Настройки для таблиц

\usepackage{amsmath} % Математические формулы
\usepackage{amsfonts} % Дополнительные математические шрифты
\makeatletter
\newcommand*{\rom}[1]{\expandafter\@slowromancap\romannumeral #1@} % Римские цифры
\makeatother

\usepackage{geometry} % Поля страницы
\geometry{
    left=20mm,
    right=10mm,
    top=20mm,
    bottom=20mm
}
\setlength{\parindent}{1.25cm} % Отступ первой строки абзаца

\usepackage{setspace} % Интерлиньяж
\onehalfspacing % Полуторный интервал

\frenchspacing % Одинаковые пробелы между словами и предложениями
\usepackage{indentfirst} % Красная строка для первого абзаца

\usepackage{titlesec} % Форматирование заголовков
\titleformat{\chapter}{\LARGE\bfseries}{\thechapter}{20pt}{\LARGE\bfseries}
\titleformat{\section}{\Large\bfseries}{\thesection}{20pt}{\Large\bfseries}
\titlespacing*{\chapter}{0pt}{-30pt}{8pt}
\titlespacing*{\section}{\parindent}{*4}{*4}
\titlespacing*{\subsection}{\parindent}{*4}{*4}

\usepackage{multirow} % Объединение строк в таблицах
\usepackage{listings} % Вставка листингов кода

% Настройки для листингов кода
\lstset{
    language=C,   					% Язык программирования
    basicstyle=\small\sffamily,			% Шрифт и размер текста
    numbers=left,						% Нумерация строк слева
    stepnumber=1,						% Шаг между номерами строк
    numbersep=5pt,						% Отступ нумерации от кода
    frame=single,						% Рамка вокруг кода
    tabsize=4,							% Размер табуляции
    captionpos=t,						% Позиция заголовка
    breaklines=true,					% Перенос длинных строк
    breakatwhitespace=true,				% Переносить строки только на пробелах
    escapeinside={\#*}{*)},				% Специальные символы в комментариях
    backgroundcolor=\color{white},      % Цвет фона
}

\usepackage[unicode,pdftex]{hyperref} % Гиперссылки в PDF
\hypersetup{hidelinks}

\usepackage{csvsimple} % Работа с CSV-файлами
\newcommand{\code}[1]{\texttt{#1}} % Команда для отображения кода внутри текста

% Дополнительные пакеты из первого файла, если они необходимы
\usepackage{tabularx} % Расширенные возможности таблиц
\usepackage{longtable} % Длинные таблицы
\usepackage{array} % Дополнительные возможности для массивов и таблиц
\usepackage{float} % Дополнительные возможности размещения плавающих объектов
\usepackage{threeparttable} % Для подписей в таблицах
\usepackage{adjustbox} % Подгонка размеров объектов
\usepackage{ragged2e} % Выравнивание текста
\usepackage{pdfpages} % Вставка страниц PDF
\usepackage{blindtext} % Генерация текста-заполнителя

% Если требуются специальные настройки для списков
\usepackage{enumitem}
\setenumerate[0]{label=\arabic*)} % Нумерация списков арабскими цифрами
% \renewcommand{\labelitemi}{---} % Маркер для ненумерованных списков

% Если используются дополнительные математические операторы
\usepackage{mathtools}
\DeclareMathOperator{\rank}{rank} % Определение оператора ранга

% Если требуется переопределение команды \underset для использования вне математического режима
\renewcommand{\underset}[2]{\ensuremath{\mathop{\mbox{#2}}\limits_{\mbox{\scriptsize #1}}}}

% Если используется команда \ulinetext из первого файла
\usepackage{xparse}
\NewDocumentCommand{\ulinetext}{O{3cm} O{c} m m}{
    \underset{#3}{\uline{\makebox[#1][#2]{#4}}}
}