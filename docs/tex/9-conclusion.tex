\begin{center}
    \textbf{ЗАКЛЮЧЕНИЕ}
\end{center}
\addcontentsline{toc}{chapter}{ЗАКЛЮЧЕНИЕ}

В ходе выполнения курсовой работы поставленная цель была достигнута: было разработано программное обеспечение для моделирования клетчатой площадки с лунками, соответствующими трёхмерным объектам, с возможностью генерации самих объектов и их падения на площадку.

Для достижения были решены следующие задачи:
\begin{itemize}
    \item[$-$] описан список доступных к размещению на сцене моделей и формализованы эти модели;
    \item[$-$] проведён анализ существующих алгоритмов компьютерной графики для визуализации сцены и выбраны наиболее подходящие;
    \item[$-$] выбраны среда реализации программного обеспечения;
    \item[$-$] разработано программное обеспечение и реализованы выбранные алгоритмы визуализации;
    \item[$-$] проведены замеры временных характеристик разработанного программного обеспечения.
\end{itemize}

\vspace{5mm}

Реализованную программу можно усовершенствовать за счет:
\begin{itemize}
    \item[$-$] распараллеливания алгоритмов для ускорения работы;
    \item[$-$] использования графической библиотеки OpenTK для повышения производительности графической визуализации путем задействования ресурсов графического процессора;
\end{itemize}